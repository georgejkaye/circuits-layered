\documentclass[10pt, a4paper]{article}
\usepackage[margin=1in]{geometry}
\usepackage[english]{babel}

\usepackage{tgpagella}

\usepackage{thmtools}

\usepackage{microtype}
\usepackage{csquotes}

\PassOptionsToPackage{usenames, dvipsnames}{xcolor}
\usepackage{figures/tikzit}

\usepackage[style=alphabetic]{biblatex}
\usepackage[colorlinks=true, citecolor=Green, linkcolor=NavyBlue, urlcolor=BrickRed]{hyperref}
\usepackage[capitalise]{cleveref}

\setcounter{biburlnumpenalty}{1000}
\setcounter{biburlucpenalty}{1000}
\setcounter{biburllcpenalty}{1000}

\declaretheorem{theorem}
\declaretheorem[sibling=theorem]{definition}
\declaretheorem[style=myremark, sibling=theorem]{example}

\addbibresource{refs/refs.biblatex.bib}

\title{String diagrams for layered explanations for digital circuits}
\author{George Kaye}
\date{}

\begin{document}

    \maketitle

    `String diagrams for layered explanations' were presented in~\cite{lobski2022string} using the formulation \emph{layered props}.
    This allows us to present a particular framework at multiple layers of abstraction.
    This lends itself very well to digital circuits, as we often use `off-the-shelf' components that model some function without needing to know how they are constructed inside.
    In fact, restricting to one internal construction would be unnecessarily limiting, as there are multiple ways a component can be built which may be more or less beneficial in certain situations.
    On another note, if diagrams were specified entirely at the gate (or transistor!) level, they would quickly get cluttered and unreadable.
    Layered props allow us to present this formally.

    \printbibliography
\end{document}